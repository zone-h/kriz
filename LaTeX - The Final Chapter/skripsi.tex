\documentclass[
10pt, % The default document font size, options: 10pt, 11pt, 12pt
%oneside, % Two side (alternating margins) for binding by default, uncomment to switch to one side
english, % ngerman for German
singlespacing, % Single line spacing, alternatives: onehalfspacing or doublespacing
%draft, % Uncomment to enable draft mode (no pictures, no links, overfull hboxes indicated)
%nolistspacing, % If the document is onehalfspacing or doublespacing, uncomment this to set spacing in lists to single
%liststotoc, % Uncomment to add the list of figures/tables/etc to the table of contents
%toctotoc, % Uncomment to add the main table of contents to the table of contents
%parskip, % Uncomment to add space between paragraphs
%nohyperref, % Uncomment to not load the hyperref package
headsepline, % Uncomment to get a line under the header
%chapterinoneline, % Uncomment to place the chapter title next to the number on one line
%consistentlayout, % Uncomment to change the layout of the declaration, abstract and acknowledgements pages to match the default layout
]{skripsiSTKIP}
\usepackage[style=apa]{biblatex}
\DeclareLanguageMapping{bahasai}{american-apa}
\addbibresource{example.bib} 

\usepackage[autostyle=true]{csquotes}

%----------------------------------------------------------------------------------------
%	Informasi Mengenai Skripsi
%----------------------------------------------------------------------------------------
\thesistitle[panduan]{Judul Skripsi}
\pembimbingPertama{Pembimbing I} 
\nikPembimbingPertama{1234567890}
\pembimbingKedua{Pembimbing II}
\nikPembimbingKedua{1234567890}

\pengujiPertama{Penguji I}
\nikPengujiPertama{1234567890}
\pengujiKedua{Penguji II}
\nikPengujiKedua{1234567890}
\pengujiKetiga{Pembimbing I}
\nikPengujiKetiga{123456789}
\pengujiKeempat{Pembimbing II}
\nikPengujiKeempat{123456789}

\kaProdi{Agus Purwanto, Ph.D.}
\nikKaProdi{0329066203}
\examiner{}
\degree{Sarjana Pendidikan (S.Pd.)} 
\penulis{Nama Kalian}
\nim{NIM Kalian}
\addresses{Tangerang}
\tahun{2017}
\bulan{Bulan Sidang}
\tanggal{Tanggal Sidang}
\hari{Hari Sidang}

\prodi{Pendidikan Fisika}
\kataKunci{Kata Kunci Abstrak Kalian: fisika gasing, latex, ibna, fismatlan, dll.}
\institusi{\href{http://www.stkipsurya.ac.id/}{Sekolah Tinggi Keguruan dan Ilmu Pendidikan Surya}}
\department{\href{http://department.university.com}{Program Studi Pendidikan Matematika}}
\group{\href{http://researchgroup.university.com}{Research Group Name}}
\faculty{\href{http://faculty.university.com}{Faculty Name}}

\AtBeginDocument{
\hypersetup{pdftitle=\ttitle}
\hypersetup{pdfauthor=\penulis}
\hypersetup{pdfkeywords=\kataKunciTulis}
}

\DefineBibliographyStrings{english}{%
  bibliography = {Daftar Pustaka},
}


\begin{document}

\frontmatter

\pagestyle{plain} 

%----------------------------------------------------------------------------------------
%	Halaman Awal
%----------------------------------------------------------------------------------------
\halamanJudul
\begin{pengesahan}

%\includegraphics[width=0.25\textwidth]{logoSTKIP-transparent.png} \hfill 
  %{\fontsize{10}{12}\fontfamily{phv} \bfseries Pengesahan \par}\hrule \vspace{4mm}

%  \chapter*{Pengesahan}
  
{ \Large \ttitle\par} % Thesis title
\vfill

{ Disetujui dan Disahkan oleh:}\\ %[5mm]
%Tangerang, \ldots\ldots\ldots\ldots\ldots \hfill
%Tangerang, \ldots\ldots\ldots\ldots\ldots\\
{ Pembimbing Pertama}
\hfill {  Pembimbing Kedua}\\ 
[18mm]


{  \underline{\pembimbingPertamaTulis}}
\hfill { \underline{\pembimbingKeduaTulis}}\\[0.8mm]
{ NIDN \nikPembimbingPertamaTulis}
\hfill { NIDN \nikPembimbingKeduaTulis}\\[0.8mm]

\vfill
{ Mengetahui:}\\ 
%Tangerang, \ldots\ldots\ldots\ldots\ldots\\
{ Ketua Program Studi Pendidikan Matematika} \\ [18mm]


{ \underline{\kaProdiTulis}}\\[0.8mm]
{ NIDN \nikKaProdiTulis} \\





\end{pengesahan}


\begin{persetujuan}

  
\noindent Skripsi saya yang berjudul \textbf{``\ttitle''} ini telah dipertahankan di depan dewan penguji pada hari $\;$\hariTulis, $\;$tanggal $\;$\tanggalTulis  $\;$ \bulanTulis $\;$\tahunTulis $\;$
dan dinyatakan diterima sebagai salah satu syarat untuk memperoleh gelar sarjana
pendidikan Program Studi Pendidikan STKIP Surya Tangerang.
\vfill

\begin{center}
\textbf{DEWAN PENGUJI}\\[5mm]

\begin{widetable}{\columnwidth}{rlc}
\textbf{Nama} &
\textbf{Jabatan} &
\textbf{Tanda Tangan}  \\  \\ \\
\underline{\pengujiPertamaTulis} & Penguji I & \ldots\ldots\ldots\ldots\ldots\ldots\ldots\ldots
\\
NIDN \nikPengujiPertamaTulis &&\\ \\ 
\underline{\pengujiKeduaTulis} & Penguji II & \ldots\ldots\ldots\ldots\ldots\ldots\ldots\ldots
\\
NIDN \nikPengujiKeduaTulis &&\\ \\
\underline{\pengujiKetigaTulis} & Penguji III & \ldots\ldots\ldots\ldots\ldots\ldots\ldots\ldots
\\
NIDN \nikPengujiKetigaTulis &&\\
\underline{\pengujiKeempatTulis} & Penguji IV & \ldots\ldots\ldots\ldots\ldots\ldots\ldots\ldots
\\
NIDN \nikPengujiKeempatTulis &&\\
\end{widetable}


\vfill
 Mengetahui:\\ 
 Ketua Program Studi \prodiTulis \\ [18mm]


 \underline{\kaProdiTulis} \\
 NIDN \nikKaProdiTulis \\

\end{center}



\end{persetujuan}



\begin{declaration}
%\addchaptertocentry{\authorshipname} % Add the declaration to the table of contents

%\chapter*{Pernyataan Keaslian Skripsi}

\noindent Saya yang bertanda tangan di bawah ini, mahasiswa Program Studi \prodiTulis, \institusiTulis:\\[3mm]
\noindent
%\begin{tabular}{@{}l@{$\:$:$\;$}l@{}}
\begin{widetable}{\textwidth}{*{3}{l}} 
Nama &:& \penulisTulis \\
Nomor Induk Mahasiswa &:& \nimTulis \\
\end{widetable}\\ [3mm]
dengan ini menyatakan bahwa: \\
Skripsi yang saya buat berjudul ``\ttitle'' adalah:
\begin{enumerate}
\item Dibuat dan diselesaikan sendiri berdasarkan hasil diskusi dengan dosen pembimbing yang
ditunjuk oleh \institusiTulis, serta ditunjang dengan hasil penelitian dan kajian pustaka yang
tertera di dalam referensi pada skripsi saya.
\item Bukan merupakan duplikasi karya tulis orang lain, kecuali dengan mencantumkan referensi yang
digunakan.
\end{enumerate}
Jika terbukti saya tidak memenuhi apa yang telah dinyatakan di atas, maka skripsi ini batal dan gelar
sarjana saya dicabut. \\ \vfill

\begin{center}
Tangerang, 20 Januari 2017\\ 
Yang membuat pernyataan \\ [18mm]

 \underline{\penulisTulis} \\[0.8mm]
  NIM. \nimTulis\\
\end{center}


\end{declaration}

%\cleardoublepage


%----------------------------------------------------------------------------------------
%	ABSTRACT PAGE
%----------------------------------------------------------------------------------------

\begin{abstract}

\vfill
\noindent Kata Kunci: \kataKunciTulis\\
Referensi: 45 (2001-2016)\\
(xviii +59; 26 tabel; 12 gambar; 45 lampiran)\\
\end{abstract}


%\begin{apresiasi}
\chapter*{Kata Pengantar}
\addchaptertocentry{Kata Pengantar} % Add the declaration to the table of contents

Peneliti berbesar hati menerima segala jenis kritikan dan saran yang bersifat membangun demi perbaikan ke arah kesempurnaan. Semoga skripsi ini dapat bermanfaat bagi peneliti lainnya sebagai bahan referensi dan bacaan.

\vfill
\begin{tabular}{lc}
\begin{minipage}{0.5\textwidth}
$\;$
\end{minipage}&
\begin{minipage}{0.4\textwidth}
\centering
Tangerang, 20 Januari 2017\\
Peneliti\\[18mm]
\penulisTulis\\
\nimTulis
\end{minipage}
\end{tabular}


\tableofcontents

\listoffigures

\listoftables

\mainmatter

\pagestyle{thesis} 

\chapter{Pendahuluan}
% Isi dari Section Sesuaikan Sendiri dengan Jenis Penelitian Kalian
\section{Latar Belakang Masalah}
\section{Batasan Masalah}
\section{Rumusan Masalah}
\section{Tujuan Penelitian}
\section{Manfaat Penelitian}
\section{Definisi Operasional}

\chapter{Kajian Pustaka}
% Isi dari Section Sesuaikan Sendiri dengan Jenis Penelitian Kalian
\section{Kajian Teori}
\section{Peta Pengembangan Produk}
\section{Kajian Penelitian yang Relevan}
\section{Tujuan Penelitian}
\section{Kerangka Berpikir}

\chapter{Metode Penelitian}
% Isi dari Section Sesuaikan Sendiri dengan Jenis Penelitian Kalian
\section{Model Pengembangan}
\section{Prosedur Pengembangan}
\section{Teknik dan Instrumen Pengumpulan Data}
\section{Teknik Analisis Data}

\chapter{Analisis Temuan dan Pembahasan} %Isi dari Chapter Bab IV Sesuaikan Sendiri dengan Jenis Penelitian Kalian
% Isi dari Section Sesuaikan Sendiri dengan Jenis Penelitian Kalian
\section{Hasil Penelitian}
\section{Proses Uji Coba Produk}
\section{Keunggulan Aplikasi}

\chapter{Penutup} %Isi dari Chapter Bab IV Sesuaikan Sendiri dengan Jenis Penelitian Kalian
% Isi dari Section Sesuaikan Sendiri dengan Jenis Penelitian Kalian
\section{Simpulan}
\section{Saran}


\appendix 

%----------------------------------------------------------------------------------------
%	Daftar Pustaka
%----------------------------------------------------------------------------------------

\printbibliography[heading=bibintoc]

\end{document}  
